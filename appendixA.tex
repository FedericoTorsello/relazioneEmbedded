\chapter{Elenco dei componenti utilizzati}

\section{Componenti hardware}
\begin{itemize}
	\item Componenti lato client:
	\begin{itemize}
		\item Arduino UNO;
		\item resistori;
		\item sensore di prossimità ad ultrasuoni HC-SR04;
		\item buzzer;
		\item potenziometro;
		\item multiplexer CD4067B;
		\item button;
		\item LED verde;
		\item LED RGB;
		\item LED rosso a 12 pin;
		\item LED giallo a 12 pin;
		\item breadboard;
		\item cavi di collegamento;
	\end{itemize}
\end{itemize}

\begin{itemize}
	\item Componenti lato server:
	\begin{itemize}
		\item Odroid C2;
		\item monitor LCD;
		\item potenziometro;
		\item breadboard;
		\item cavi di collegamento.
	\end{itemize}
\end{itemize}

\section{Componenti software}
\subsection{Librerie Arduino}
\begin{itemize}
	\item \href{http://playground.arduino.cc/Code/NewPing}{NewPing};
	\item \href{https://github.com/bblanchon/ArduinoJson}{ArduinoJson}.
\end{itemize}

\subsubsection{\underline{\href{http://playground.arduino.cc/Code/NewPing}{NewPing}}}\label{sec:newping}
\textbf{Caratteristiche}
\begin{itemize}
	\item Funziona con diversi modelli di sensori ad ultrasuoni: SR04, SRF05, SRF06, DYP-ME007 e Parallax Ping™;
	\item Non ha un \textbf{lag} di un secondo se non si riceve un ping di eco
	\item Ping coerente e affidabile fino a 30 volte al secondo.
	\item Timer interrupt method per sketch event-driven
	\item Metodo di filtro digitale Built-in \texttt{ping\_median()} per facilitare la correzione degli errori.
	\item Utilizzo dei registri delle porte durante l'accesso ai pin per avere un'esecuzione più veloce e dimensioni del codice ridotte.
	\item Consente l'impostazione di una massima distanza di lettura del ping "in chiaro".
	\item Facilita l'utilizzo di più sensori.
	\item Calcolo distanza preciso, in centimetri, pollici e uS.
	\item Non fa uso di \texttt{pulseIn}, in quanto lento e con alcuni modelli di sensore a ultrasuoni restituisce risultati errati.
	\item Attualmente in sviluppo, con caratteristiche che vengono aggiunte e bug/issues affrontati.
\end{itemize}

\subsubsection{\underline{\href{https://github.com/bblanchon/ArduinoJson}{ArduinoJson}}}\label{sec:arduinojason}
\textbf{Caratteristiche}
\begin{itemize}
	\item codifica/decodifica JSON;
	\item API di facile utilizzo;
	\item allocazione di memoria fissa (senza malloc);
	\item nessuna \textit{data duplication} (evitando la copia);
	\item portatile (scritto in C++98);
	\item non ha dipendenze esterne;
	\item libreria leggera;
	\item MIT License.
\end{itemize}

\subsection{IDE utilizzati}
\begin{itemize}
	\item \href{https://atom.io/}{Atom} con \href{ http://platformio.org/}{PlatformIO};
	\item \href{https://www.arduino.cc/en/Main/Software}{Arduino IDE} (per alcuni test sulla comunicazione seriale).
\end{itemize}

\subsection{Linguaggi di sviluppo}
\begin{itemize}
	\item Wiring/C++;
	\item Python;
	\item JSON;
	\item HTML;
	\item PHP;
	\item Javascript/AJAX.
\end{itemize}

\subsection{Altro}
\begin{itemize}
	\item \href{http://fritzing.org}{Fritzing} per riportare gli schemi di collegamento.
\end{itemize}