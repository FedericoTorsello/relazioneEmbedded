\chapter{Elenco dei componenti utilizzati}

\section{Componenti hardware}
\begin{itemize}
	\item Componenti lato client:
	\begin{itemize}
		\item Arduino UNO
		\item Breadboard
		\item Cavi di collegamento
		\item Resistori
		\item Sensore di prossimità ad ultrasuoni HC-SR04
		\item Buzzer
		\item Potenziometro
		\item Multiplexer CD4067B
		\item Button
		\item LED verde
		\item LED RGB
		\item LED rosso 6 pin
		\item LED giallo 6 pin
	\end{itemize}
\end{itemize}

\begin{itemize}
	\item Componenti lato server:
	\begin{itemize}
		\item Odroid C2
		\item Monitor LCD
	\end{itemize}
\end{itemize}

\section{Componenti software}
Librerie Arduino:
\begin{itemize}
	\item \href{http://playground.arduino.cc/Code/NewPing}{NewPing}
	\item \href{https://github.com/bblanchon/ArduinoJson}{ArduinoJson} 
\end{itemize}
IDE utilizzati: 
\begin{itemize}
	\item \href{https://atom.io/}{Atom} con \href{ http://platformio.org/}{PlatformIO} 
	\item \href{https://www.arduino.cc/en/Main/Software}{Arduino IDE} (per alcuni test sulla comunicazione seriale) 
\end{itemize}
Linguaggi di sviluppo:
\begin{itemize}
	\item Wiring/C++, per lo sketch Arduino lato client 
	\item Python
	\item JSON per la codifica della comunicazione USB e remota
	\item HTML ...
\end{itemize}
Altro:
\begin{itemize}
	\item \href{http://fritzing.org}{Fritzing} per lo schema di collegamento
\end{itemize}