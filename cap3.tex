\chapter{Giocare a Jimmy Challenge}
\section{Giocare con la mano e con i sensi}
L'obiettivo del giocatore è trovare la giusta posizione dei pistoncini e quindi scassinare il lucchetto nel minor tempo possibile. Questo gioco è sia un \textbf{multiplayer online} che un \textbf{single game offline}.

La posizione dei pistoncini viene assegnata in modo random ad ogni nuovo livello e rimane fissa fino al suo superamento.

Per trovare la posizione attuale dei pistoncini al giocatore basta muovere la mano orizzontalmente in direzione del sensore ad ultrasuoni. (La rilevazione del lucchetto è spiegata più avanti).

Durante le varie fasi di gioco l'utente ha la possibilità di rendersi conto dell'evoluzione del gioco \textbf{ascoltando i suoni} emessi dal buzzer e/o \textbf{guardando i colori} dei LED.

\subsection{Significato dei suoni e dei colori}
\begin{itemize}
	\item All'avvio del gioco:
	\begin{itemize}
		\item i LED a 12 pin giallo e rosso fanno un carosello, cioè si alternano nell'accensione/spegnimento;
		\item il LED verde e il buzzer si comportano come quando la corretta posizione del pistoncini non è stata trovata.
	\end{itemize}
\end{itemize}

\begin{itemize}
	\item Quando \textbf{non} si è trovato il pistoncino:
	\begin{itemize}
		\item il LED verde emette una luce pulsante;
		\item il LED RGB emette una luce continua di color blu chiaro;
		\item il buzzer suona due note in modo frenetico.
	\end{itemize}
\end{itemize}

\begin{itemize}
	\item Quando si è nell'area della posizione corretta del pistoncino:
	\begin{itemize}
		\item il LED verde emette una luce fissa;
		\item il LED RGB continuerà ad emettere una luce continua blu chiaro, ma solo fino a quando non entrerà nello \textbf{stato di scasso};
		\item il buzzer suona due note meno freneticamente.
	\end{itemize}
\end{itemize}

\subsection{Superare un livello}
Per superare il livello il ladro deve forzare il lucchetto trovando la giusta posizione dei pistoncini.

Dal punto di vista del giocatore ogni livello rappresenta un lucchetto da aprire e la posizione del pistoncino che compone ogni lucchetto è rappresentata dal range di spazio posta davanti al sensore (in linea orizzontale). Una volta trovata la corretta posizione inizierà il processo di forzatura e di scasso del lucchetto e non si deve muovere per non rischiare di perdere la posizione del pistoncino.

Per forzare il lucchetto è sufficiente tenere una mano davanti al sensore ad ultrasuoni per un tempo limitato, avviando lo stato di scasso. Se il tempo di scasso non viene rispettato o la mano viene rimossa troppo presto, il livello riparte senza salvare i progressi.

\begin{itemize}
	\item \textbf{Non} si supera il livello:
	\begin{itemize}
			\item Se la mano viene spostata dall'area corretta del pistoncino troppo presto, per esempio non si è ancora nello stato di scasso
			\item Se si rimane troppo tempo nella stato di scasso (il grimaldello è stato "rotto").
	\end{itemize}
\end{itemize}

\begin{itemize}
	\item Si può superare il livello:
	\begin{itemize}
			\item Se la mano resta fissa nella posizione in cui si trova il pistoncino, rispettando il tempo dello \textbf{stato di scasso} (\ref{sec:statodiscasso}) e poi la si agita sempre nell'area del lucchetto per simulare il processo di apertura.
	\end{itemize}
\end{itemize}

\clearpage
\subsection{Stato di scasso}\label{sec:statodiscasso}
Se si rispetta il tempo nello stato di scasso e quindi si apre il lucchetto, si supera il livello.

Per indicare lo stato di scasso si è utilizzato un LED RGB. 
\begin{itemize}
	\item Ogni colore ha un significato:
	\begin{itemize}
		\item blu scuro: si sta iniziando a scassinare il lucchetto
		\item verde: il lucchetto è scassinato;
			\subitem NB: per passare al livello successivo si deve togliere la mano e riposizionarla nell'area del pistoncino [come se si girasse la "\textit{chiave}"].
		\item arancio: attenzione, se non si toglie la mano ora si rischia di rompere il grimaldello;
		\item giallo: pericolo di rottura ancora più elevato;
		\item rosso: il grimaldello è stato rotto, quindi il livello deve essere ricominciato.
	\end{itemize}	
\end{itemize}
