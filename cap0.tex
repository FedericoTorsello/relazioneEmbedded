\begin{abstract}

In questa relazione si descriverà come è stato progettato il gioco interattivo \textbf{Jimmy Challenge}, descrivendo le scelte di progettuali e le fasi implementative.

Jimmy Challenge è stato realizzato utilizzando diversi componenti hardware/software, con l'obiettivo di sfruttare ed ottimizzare quanto più possibile la board Arduino in un'ottica IoT.

\begin{description}
	\item [Capitolo 1] In questo capitolo si introduce il progetto e l'interazione giocatore-sistema.
	\item [Capotolo 2] Si descrivono i requisiti, i casi d'uso, il modeling, il design e l'analysis del sistema.
	\item [Capitolo 3] In questo capitolo si illustra come un giocatore si interfaccia con Jimmy Challenge, descrivendo il significato dei suoni, dei colori, cosa si intende per stato di scasso e quindi come si fare per superare un livello.
	\item [Capitolo 4] Descrizione delle classi astratte e dei relativi metodi virtuali.
	\item [Capitolo 5] Descrizione ed analisi delle classi concrete e dei relativi metodi.
	\item [Capitolo 6] In questo capitolo si descrive la realizzazione dei task in Arduino, il multi-tasking dal punto di vista del progettista e dell'utente e le scelte progettuali inerenti ai task.
	\item [Capitolo 7] Capitolo dedicato alla parte alla sicurezza dell'IoT, le scelte progettuali lato server e le tecnologie Web utilizzate.
	\item [Capitolo 8] Testing del progetto, verificando che il gioco risponde a tutti i requisiti descritti nel progetto.	
	\item [Capitolo 9] Conclusioni finali.
\end{description}


\end{abstract}
