\chapter{Introduzione}

\textbf{Jimmy Challenge} è un gioco interattivo in cui un "ladro" tenta di \textit{forzare} un lucchetto utilizzando un \textit{grimaldello} [jimmy in inglese]. 

Per poter giocare bisogna alimentare l'Arduino e attendere qualche secondo di setup.

Una volta che il settaggio è completo, l'utente può interagire con Arduino utilizzando diversi componenti hardware che mutano il loro comportamento in base al contesto.

\section{Giocare}
L'obiettivo del giocatore è trovare e quindi scassinare il lucchetto nel minor tempo possibile. Questo gioco è giocabile online (uno contro uno) che offline.

La posizione del lucchetto viene assegnata in modo random ad ogni nuovo livello e rimane fissa fino al suo superamento.

Per trovare la posizione attuale del lucchetto bisogna muovere la mano orizzontalmente in direzione del sensore ad ultrasuoni.

Durante le varie fasi di gioco l'utente ha la possibilità di rendersi conto dell'evolvuzione del gioco ascoltando i suoni emessi dal buzzer o guardando i colori dei LED.

\subsection{Significato dei suoni e dei colori}
\begin{itemize}
	\item All'avvio del gioco
	\begin{itemize}
		\item i LED a 12 pin giallo e rosso fanno un carosello;
		\item il LED verde e il buzzer si comportano come quando il lucchetto non è stato trovato.
	\end{itemize}
\end{itemize}


\begin{itemize}
	\item Quando \textbf{non} si è trovato il lucchetto:
	\begin{itemize}
		\item il LED verde emette una luce pulsante;
		\item il LED RGB emette una luce continua di color blu chiaro;
		\item il buzzer suona due note in modo frenetico.
	\end{itemize}
\end{itemize}

\begin{itemize}
	\item Quando si è nell'area del lucchetto:
	\begin{itemize}
		\item il LED verde emette una luce fissa;
		\item il LED RGB continuerà ad emettere una luce continua blu chiaro, ma solo fino a quando non entrerà nello stato di scasso;
		\item il buzzer suona due note meno freneticamente.
	\end{itemize}
\end{itemize}

\subsection{Superare un livello}
Per superare il livello il ladro deve forzare il lucchetto.

Dal punto di vista del giocatore il lucchetto è un'area nello spazio posta davanti al sensore (in linea orizzontale).

Per forzare il lucchetto è sufficiente utilizzare una mano posta davanti al sensore ad ultrasuoni per un tempo delimitato, avviando lo stato di scasso. Se il tempo di scasso non viene rispettato o la mano viene rimossa troppo presto, il livello riparte senza salvare i progressi.

\begin{itemize}
	\item \textbf{Non} si supera il livello:
	\begin{itemize}
			\item Se la mano viene spostata dall'area del lucchetto troppo presto, per esempio non si è ancora nello stato di scasso
			\item Se si rimane troppo tempo nella fase di scasso (il lucchetto è stato "rotto").
	\end{itemize}
\end{itemize}

\begin{itemize}
	\item Si può superare il livello:
	\begin{itemize}
			\item Se la mano resta fissa nella posizione in cui si trova il lucchetto, rispettando il tempo nello \textbf{stato di scasso} e poi la si agita sempre nell'area del lucchtto ("aprendolo").
	\end{itemize}
\end{itemize}

\subsection{Stato di scasso}
Se si rispetta il tempo nello stato di scasso e quindi si apre il lucchetto, si supera il livello.

Per indicare lo stato di scasso si è utilizzato il LED RGB. 
\begin{itemize}
	\item Ogni colore ha un significato:
	\begin{itemize}
		\item blu scuro: si sta scassinando il lucchetto
		\item verde: il lucchetto è scassinato.
			\subitem NB: per passare al livello successivo si deve togliere la mano e riposizionarla nell'area del lucchetto [come se si infilasse la "\textit{chiave}"].
		\item arancio: attenzione, se non si toglie la mano ora si rischia di rompere il lucchetto
		\item giallo: pericolo di rottura ancora più elevato
		\item rosso: il lucchetto è stato rotto, quindi il livello deve essere ricominciato di nuovo.
	\end{itemize}
\end{itemize}

Task:
\begin{itemize}
	\item SonarTask
	\item ButtonTask
	\item BuzzerTask
	\item LedTask
	\item LedPwmTask
	\item LedRgbTask
\end{itemize}