\chapter{Introduzione}
\textbf{Jimmy Challenge}\footnote{"\textit{jimmy}" in inglese vuol dire grimaldello, da questo il nome del gioco.} è un gioco interattivo che si ispira all'attività di \textit{lock-picking}: aprire un lucchetto o una serratura usando ad esempio un \textbf{grimaldello} per manipolare i pistoncini interni per simulare la presenza della chiave originale.

Per poter giocare è necessario alimentare l'\href{https://www.arduino.cc/en/Main/ArduinoBoardUno}{Arduino UNO\footnote{Arduino UNO - è una scheda elettronica di piccole dimensioni con un microcontrollore ATmega, utile per creare rapidamente prototipi e per scopi hobbistici, didattici e professionali. (\url{https://it.wikipedia.org/wiki/Arduino\_\%28hardware\%29})}} e attendere qualche secondo di setup.

Una volta che il settaggio è completo, l'utente può interagire con Arduino utilizzando diversi componenti hardware che mutano il loro comportamento in base al contesto.




