\section{Networking e Sicurezza}

Il mondo embedded e IoT che si sta sviluppando in questi anni, grazie anche alla presenza sempre maggiore di dispositivi dal rapporto prestazioni/prezzo vantaggioso, avvicina molti \textit{inesperti} al mondo dell'elettronica/informatica, permettendo una rapida diffusione e quindi lo sviluppo di nuove librerie e sistemi software orientati all'IoT. Questo \textit{trend} si sta materializzando con l'avvento di in un nuovo mercato sempre più vasto, composto da produttori di device dalle dimensioni ridotte che risultano essere performanti, economici, perfetti per realizzazioni embedded.

A causa di questo sviluppo, quasi incrontrollato, in cui tutto può e viene  collegato \textit{online}, si è persa la visione della \textbf{sicurezza dei sistemi} (hardware e software). I primi attacchi a questi sistemi hanno avuto facilità di esecuzione e una rapidissima diffusione (una breve lista di attacchi a sistemi IoT nel 2015 \url{http://tinyurl.com/honlko2}).

In questo scenario, dalla parte dei produttori hardware si registra una mancanza o addirittura una resistenza nel supporto nella correzione di falle o al rilascio di aggiornamenti firmware, a causa di \textbf{codice proprietario non modificabile} o economicamente svantaggioso per l'azienda stessa. Per certi versi si potrebbe parlare di \textbf{obsolescenza programmata}.

Anche da parte degli utilizzatori dei sistemi finali, non si evince una particolare attenzione ai problemi relativi alla sicurezza a causa di scarse conoscenze del sistema in uso o del \textit{closed source} dello stesso. È tristemente noto che lo sviluppo di questi dispositivi \textit{perennemente connessi}, non prende quasi mai in considerazione le problematiche relative alla sicurezza derivata dalla connessione a sistemi più \textit{fragili} o legate all'interazione con altri device (perdita di privacy o di prestazioni, utilizzo improprio di risorse di rete).

Jimmy Challenge è stato sviluppato garantendo la sicurezza del sistema e degli utenti che lo utilizzano. In questo capitolo verranno presentate alcune delle soluzioni adottate e le tecnologie utilizzate per l'implementazione dell'applicativo lato server.
In generale si è preferito utilizzare esclusivamente software \textit{open-source} (ove possibile).

\subsection{Sistema Operativo}
Il server (fisicamente un \textbf{Odroid C2}) esegue una distribuzione Arch Linux ARM a 64-bit. Questa distro, più \textit{bleeding edge} per quanto riguarda l'upstream degli aggiornamenti, è sempre aggiornata all'ultima versione dei software disponibile, generalmente considerata più sicura, e fornisce una libertà maggiore per quanto riguarda la configurazione del sistema stesso.

\subsubsection{Software orientato alla sicurezza installato}
\begin{itemize}
	\item SSH con autenticazione chiave pubblica-privata RSA;
	\item UFW come firewall per rendere disponibili soltanto alcuni servizi dall'esterno:
	\begin{itemize}
		\item 80: Web Server;
		\item 22: SSH;
		\item 443: HTTPS.
	\end{itemize}
	\item Cerbot (https://certbot.eff.org/) per i certificati SSL/TLS;
	\item Netdata (https://github.com/firehol/netdata) per una visualizzazione delle risorse del sistema da remoto.
\end{itemize}
Oltre all'installazione e alla configurazione dei software sono stati configurati diversi parametri per l'\textit{hardening} del kernel tramite \textit{sysctl} assieme ad un sistema di \textit{logging} per registare gli accessi.

\subsection{NGINX e HTTPS}
\textbf{NGINX} è un web server orientato alle performance, al ridotto consumo di risorse e alla facilità di configurazione.
In un sistema come l'Odroid con potenza computazionale ridotta (rispetto ai classici web server) l'utilizzo di applicativi con ridotto consumo di risorse garantisce una maggiore reattività del sistema e un migliore risulto agli occhi dell'utente.

La configurazione di NGINX prende in considerazione tutti i maggiori problemi legati all'esposizione di un sito e di un web server in rete: buffer overflow, DDoS, bruteforcing, sniffing, etc. e cerca di mitigare o prevenire ogni tipo di attacco senza generare intralcio o rallentamenti all'utilizzo standard del web server.
[link configuazione NGINX]
Grazie a \textit{Cerbot} è possibile ottenere certificati SSL/TLS gratuitamente e garantire un canale di comunicazione protetto tra client e server.

L'intera configurazione di NGINX cerca di essere compatibile con gli ultimi protocolli e limitare quelli obsoleti e poco sicuri, rischiando però di ridurre la compatibilità con il browser. Per garantire ulteriori performance NGINX è abilitato ad instaurare connessioni tramite il recente HTTP/2 con i client compatibili.

\subsection{MySQL, Redis}
Con \textbf{MySQL} si gestisce il database per lo storage delle credenziali degli utenti e la lista degli utenti connessi online.
[link dump sql]
Un event MySQL controlla periodicamente se un utente è ancora collegato o meno.

\textbf{Redis} è un applicativo per lo store di strutture dati chiave-valore in memoria. Il principale vantaggio di Redis è la sua velocità in lettura e scrittura e la facilità di utilizzo.
Nè MySQL nè Redis sono esposti verso l'esterno ma entrambi sono protetti da password e configurati per rispettare il POLP (\textit{principle of least privilege}).

\subsection{Sito Web, back-end e API RESTful}
Il sito web (sia front-end che back-end) è stato sviluppato tenendo a mente le linee guida del W3C e del OWASP. Da test dedicati non presenta lampanti vulnerabilità come XSS (\textit{Cross-site scripting}), \textit{clickjacking} o \textit{session hijacking}.

Al fine di rendere l'applicativo il più possibile portabile, sono state implementate delle API RESTful in PHP7; sia l'applicativo Python lato client, sia il sito web (tramite AJAX) utilizzano queste API.

Una volta che un utente si è loggato, gli viene assegnato un JWT (JSON Web Signature) che deve essere validato ad ogni richiesta che farà l'utente da quel momento in poi. Questo \textit{token} permette di verificare l'identità dell'utente ad ogni richiesta.
Le API possono essere invocate solo tramite richieste POST e comunicano tramite messaggi JSON appositamente formattati.

In seguito al login effettuato, il sito web si presenta all'utente come un'unica pagina in cui è possibile inviare messaggi a tutti gli altri utenti collegati tramite una chat globale e selezionare un giocatore da sfidare. 

Una volta iniziata una partita, nella view principale della pagina appariranno i dati relativi allo stato della partita dell'avversario. Ogni aggiornamento dello stato verso i client è effettuato tramite Server-Sent Events (SSE).

Grazie ai SSE, per mezzo di pattern \textbf{publish-subscribe}:
\begin{itemize}
	\item si riducono i consumi di risorse computazionali
	\item si limita l'utilizzo della banda
	\item si eliminano i problemi di gestione di altre tecnologie come WebSocket o WebRTC.
\end{itemize} 
Per farlo si instaura un canale mono direzionale verso il client, attivo solo in presenta di nuovi dati.

Per il back-end sono state utilizzate diverse librerie per fornire alcune funzionalità come i JWS e i SSE scelte appositamente per la loro compatibilità con gli ultimi standard e versioni di PHP al fine di garantire maggiori performance e sicurezza:
\begin{itemize}
	\item \textbf{jose} (\url{https://github.com/namshi/jose}) per i JWS;
	\item \textbf{libSSE-php} (\url{https://github.com/licson0729/libSSE-php}) per SSE;
	\item \textbf{http-foundation} (\url{https://github.com/symfony/http-foundation})
	\item \textbf{predis} (\url{https://github.com/nrk/predis}) come dipendenze di "libSSE-php".
\end{itemize}
Le interazioni tramite il database MySQL avvengono per mezzo di una connessione locale, utilizzando le varie tecniche di mitigazione per SQLi di primo e secondo livello fornite dal PHP.

Per proteggere la privacy degli utenti, le password sono salvate sul database come hash (usando bcrypt).

\subsection{Python}
Come anticipato nella relazione, per poter inviare al server i dati letti dalla seriale di Arduino è stato sviluppato uno script Python, in particolare \textbf{Python 3.5}.

Tenendo a mente la visione di utilizzare sistemi embedded e quindi dalle capacità computazionali ridotte rispetto ad un laptop o ad un PC desktop, si è deciso di utilizzare:
\begin{itemize}
	\item \textbf{pyserial} (\url{https://github.com/pyserial/pyserial}) : per la lettura dei dati dalla seriale;
	\item \textbf{ujson} (\url{https://github.com/pyserial/pyserial}) : per la i messaggi JSON;
	\item \textbf{asyncio} (\url{https://github.com/python/asyncio}) : per eseguire le routine in maniera asincrona;
	\item \textbf{aiohttp} (\url{https://github.com/KeepSafe/aiohttp}) : come client per asyncio.
\end{itemize}

Per poter utilizzare lo script è necessario autenticarsi e ricevere il token JWS al fine di effettuare le richieste HTTP.

L'obiettivo di rendere completamente asincrono il codice ed eliminare il thread della libreria pyserial teoricamente sarebbe possibile utilizzando anche delle API per asyncio, ma al momento dello sviluppo del progetto queste soluzioni sono incomplete e non affidabili, quindi non sono state scartate.

\subsubsection{Pyserial}
La libreria \textbf{pyserial} consente la connessione all'Arduino tramite porta seriale e la lettura dei messaggi in maniera continua grazie all'utilizzo di un thread che agisce da produttore di messaggi per l'intero script.

L'elaborazione di questi messaggi viene effettuata da un routine che dopo aver \textit{parsato} il messaggio li instrada al corretto destinatario. Se si deve inviare un messaggio al server, viene composto il messaggio in JSON e viene inviato tramite una richiesta HTTP POST al server.

\subsubsection{Asyncio}
A parte il thread creato per la lettura dei messaggi da Arduino, le altre \textit{routine}, o meglio, \textit{\textbf{coroutine}} sono eseguite all'interno dell'event loop di asyncio. 

Asyncio fornisce una infrastruttura single-threaded per la scrittura di codice concorrente utilizzando coroutine, in particolare è consigliato per \textbf{programmi concorrenti IO-bound}. 

\subsubsection{Aiohttp}
Aiohttp fornisce un supporto per richieste HTTP asincrone tramite l'infrastruttura messe a disposizione da asyncio.

Questo tipo di approccio:
\begin{itemize}
	\item permette di ridurre i consumi di risorse;
	\item aumentare notevolmente le performance dell'applicazione.
\end{itemize}
Questo risultato si ottiene perché la maggior parte dell'esecuzione dello script è basata sull'invio di dati al server.

