\chapter{Classi concrete}
\section{Classi concrete sviluppate}
\begin{itemize}
	\item Classi utilizzate per l'\textbf{input}:
	\begin{enumerate}
		\item Sonar.cpp, Sonar.h;
		\item Button.cpp, Button.h.
	\end{enumerate}
	\item Classi utilizzate per l'\textbf{output}:
	\begin{enumerate}
		\item Buzzer.cpp, Buzzer.h;
		\item Led.cpp, Led.h;
		\item LedPwm.cpp, LedPwm.h;
		\item LedRgb.cpp, LedRgb.h;
		\item MessageService.cpp, MessageService.h;
		\item Multiplexer.cpp, Multiplexer.h.
	\end{enumerate}
\end{itemize}

\subsection{Sonar.cpp, Sonar.h}
Queste classi permettono di leggere la distanza tra la mano del giocatore ed il sensore ad ultrasuoni.

Per ottenere un input molto performante dal punto di vista del tempo e dell'accuratezza, è stata sfruttata la libreria \href{http://playground.arduino.cc/Code/NewPing}{NewPing}.

In particolare nel costruttore di Sonar viene istanziato un oggetto NewPing a cui sono passati tre parametri:
\begin{itemize}
	\item \texttt{trigPin} = pin settato come output a cui è fisicamente collegato il trigger del sonar;
	\item \texttt{echoPin} = pin settato come output a cui è fisicamente collegato l'echo del sonar;
	\item \texttt{maxDistance} = limita massimo di distanza gestito oltre il quale la mano non viene rilevata e non si possono creare lucchetti.
\end{itemize}

\subsubsection{Metodi}
\begin{itemize}
	\item \texttt{int readDistance()}: lettura istantanea della distanza mano-sensore espressa in cm.
\end{itemize}

\subsection{Button.cpp, Button.h}
\subsubsection{Metodi}
\begin{itemize}
	\item \texttt{bool readBool();}
\end{itemize}

\subsection{Buzzer.cpp, Buzzer.h}
\subsubsection{Metodi}
\begin{itemize}
	\item \texttt{public void playSound(const int sound);}
	\item \texttt{private void playMarioTheme();}
	\item \texttt{private void buzz(int, int);}
\end{itemize}

\subsection{Led.cpp, Led.h}
\subsubsection{Metodi}
\begin{itemize}
	\item \texttt{protected	void switchOn();}
	\item \texttt{protected	void switchOff();}
\end{itemize}

\subsection{LedPwm.cpp, LedPwm.h}
\subsubsection{Metodi}
\begin{itemize}
	\item \texttt{public void setIntensity(uint8\_t);}
	\item \texttt{public void switchOn();}
	\item \texttt{public void switchOff();}
\end{itemize}

\subsection{LedRgb.cpp, LedRgb.h}
\subsubsection{Metodi}
\begin{itemize}
	\item \texttt{public void setColor(int, int, int);}
	\item \texttt{protected	void switchOn();}
	\item \texttt{protected	void switchOff();}
\end{itemize}

\subsection{MessageService.cpp, MessageService.h}
\subsubsection{Metodi}
\begin{itemize}
	\item \texttt{public void init(const int, const String \&);}
	\item \texttt{public void setMessage(String);}
	\item \texttt{public String getMessage();}
	\item \texttt{public void errorMsg();}
	\item \texttt{public void ackMsg(const String);}
	\item \texttt{public void sendMsg(const String, const String);}
	\item \texttt{public void sendInfo(const int, const int, const uint8\_t, const String)}.
\end{itemize}

\subsection{Multiplexer.cpp, Multiplexer.h}
\subsubsection{Metodi}
\begin{itemize}
	\item \texttt{public void switchOn(int);}
	\item \texttt{public void carouselYellow(int);}
	\item \texttt{public void carouselRed(int);}
\end{itemize}
