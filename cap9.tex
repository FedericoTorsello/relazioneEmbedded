\chapter{Conclusioni}
Il progetto realizzato è stato utile nel dimostrare le potenzialità della piattaforma Arduino che, nella sua semplicità, permette la creazione di sistemi non banali utilizzando tecnologie e linguaggi di programmazione di alto livello.\\ 
Oltre alle potenzialità sono state evidenziate anche delle limitazioni della piattaforma:
\begin{itemize}
	\item il linguaggio Wiring non permette alcuni costrutti di alto livello e non supporta alcune librerie standard del C++;
	\item la memoria è limitata e si deve fare attenzione a non saturarla per evitare \textit{lag} (o \textit{freeze}) che possono causare problemi per applicazioni real-time;
	\item il numero di pin utilizzabili è limitato e ha costituito un \textit{upper bound} al numero di funzionalità realizzate;
	\item nativamente Arduino UNO non supporta lo scambio di messaggi via rete o Bluetooth, ma soltanto via seriale.
\end{itemize}
Storicamente Arduino è stata la prima azienda a portare dei microcontrollori nel mercato di massa grazie alla sua facilità di utilizzo e una \textit{community} sempre più grande; questo ha dato inizio al crescente sviluppo del mondo embedded e alla creazione dei primi SoC e dei computer \textit{single-board} dalle dimensioni di una carta di credito.

Guardando con un occhio al futuro e considerando le tecnologie ora presenti è possibile dichiarare che ogni dispositivo sarà connesso ad Internet, o ad altri dispositivi, al fine di creare un'unica rete di \textbf{device intelligenti} al fine di garantire una vita migliore, delle condizioni di lavoro migliori, una sicurezza maggiore a tutti coloro che ne fanno parte.