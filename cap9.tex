\chapter{Conclusioni}
Il progetto realizzato è stato utile nel dimostare le potenzialità della piattaforma Arduino che, nella sua semplicità, permette la creazione di sistemi non banali utilizzando tecnologie e linguaggi di programmazione di alto livello.\\ 
Oltre alle potenzialità sono state evidenziate anche delle limitazioni della piattaforma:
\begin{itemize}
	\item il linguaggio non permette alcuni costrutti di alto livello e non supporta alcune librerie standard del C++;
	\item la memoria è limitata è si deve fare attenzione a non saturarla per evitare lag (o freeze) che possono causare problemi per applicazioni real-time;
	\item il numero di pin utilizzabili è limitato e ha costituito un upper bound al numero di funzionalità realizzate;
	\item nativamente Arduino Uno non supporta lo scambio di messaggi via rete o bluetooth ma soltanto via seriale;
\end{itemize}
Storicamente Arduino è stata la prima azienda a portare dei microcontrollori nel mercato di massa grazie alla sua facilità di utilizzo e una community sempre più grande; questo è dato inizio al grande sviluppo del mondo embedded e alla creazione dei primi SoC e dei computer single-board dalle dimensioni di una carta di credito.
Guardando con un occhio al futuro e considerando le tecnologie ora presenti è possibile dichiarare che ogni dispositivo sarà (ed è) connesso ad internet, o ad altri dispostivi, al fine di creare un'unica rete di dispostivi intelligenti attua a garantire una vita migliore, delle condizioni di lavoro migliori, una sicurezza maggiore a tutti coloro che ne fanno parte.