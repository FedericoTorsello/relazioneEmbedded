\chapter{Progettazione di un sistema Embedded}
\section{Analisi dei requisiti}
\subsection{Requisiti funzionali}
Il sistema realizzato deve permettere ad uno o due giocatori di competere per sbloccare dei lucchetti. Ad ogni lucchetto sbloccato si supera il livello fino a quando i livelli non terminano e il gioco si dice finito. Durante il gioco sono presenti delle penalità e dei bonus.

\subsection{Requisiti non funzionali}
Le performance utili per la buona riuscita del progetto a cui non si può rinunciare riguardano la rilevazione della distanza e l'invio dei feedback locali/remoti.
Il tempo gioca un ruolo importante in questo progetto, quindi si devono evitare inutili ritardi nella rilevazione ed invio delle informazioni.

\section{Modeling}
Il sistema si suddivide in più parti:
\begin{itemize}
	\item parte fisica lato client/server
	\item parte software lato client/server
	\item networking lato client/server
\end{itemize}

\subsection{Modellazione della parte fisica - lato client}
Dal punto di vista fisico, si ha un Arduino UNO:
 \begin{itemize}
 	\item connesso ad una breadboard e a dei componenti hardware attraverso i sui pin digitali
 	\item connesso ad un PC mediante la porta USB (da cui riceve l'alimentazione a 5v). 
 \end{itemize}
 Ogni componente della breadboard ha un proprio impiego distinguendoli in \textbf{sensori} ed \textbf{attuatori} secondo la visione di \textit{sistema reattivo}.

	\begin{quote}
		\textbf{Reactive system}: è l’ambiente esterno che determina gli eventi che
		condizionano l’esecuzione del sistema
	\end{quote}
	
La connessione seriale via USB serve per inviare i dati campionati dall'Arduino verso il PC. A loro volta questi dati possono servire per creare una GUI locale (per esempio su terminale) o remota, su browser.

\subsection{Modellazione della parte logica - lato client}
Per realizzare il comportamento dinamico del progettare, nella parte software si è replicato il funzionamento di una macchina a stati finiti (\textit{FSM}) che esegue dei task.

	\begin{quote}
		\textbf{Macchine a stati finiti} (o \textbf{\textit{automi a stati finiti}}): sono il modello nel discreto più utilizzato per modellare sistemi embedded.
		
		Ogni FSM opera in una sequenza di passi (discreti) e la sua dinamica è caratterizzata da sequenze di eventi (discreti).
		
		Un evento discreto avviene ad un determinato istante e non ha durata.
	\end{quote}
	
	\begin{quote}
		\textbf{Decomposizione in task}: principio di progettazione importante, rende modulare il sistema in quanto:
		\begin{itemize}
			\item ogni modulo è rappresentato da un task (compito da eseguire)
			\item un task può essere decomposto in sotto-task in modo ricorsivo o un task complesso può essere definito come composizione di sotto-task più semplici
		\end{itemize}
	\end{quote}
	
Si è fatto uso della modellazione ad oggetti, modellando ogni task come una classe separata che estende da una classe base task. In ogni task viene fatto l'\textbf{\textit{inject} del comportamento} da avere ad ogni tick.

Il multi-tasking realizzato è cooperativo, tipo round-robin in quanto c'è parallelismo e le funzioni tick() sono chiamate sequenzialmente all'interno del loop dello sheduler.

\begin{itemize}	
	\item si è fatto uso della programmazione Object-Oriented per rendere la struttura del codice ingegneristica, lineare e scalabile;
	\item si sono utilizzate le lambda expression per fare l'\textit{inject} di codice da eseguire nei vari task.
\end{itemize}

Per quanto riguarda la connessione USB, per scelta progettuale tutti i dati inviati dall'Arduino al PC sono tutti formattati in \textbf{JSON}.

\subsection{Modellazione della parte fisica - lato server}
\subsection{Modellazione della parte logica - lato server}
\section{Design}
Il sistema che si suddivide in più aree:
\begin{enumerate}
	\item I/O locale attraverso Arduino UNO
	\item feedback remoto su browser
	\item input per invio dei dati seriali al server attraverso async task Python
	\item connessione USB da Arduino UNO verso il PC
	\item connessione del PC ad un server
	\item sito internet come GUI remota
	\item gestione del server remoto
\end{enumerate}

Le tecno

– processo finalizzato alla creazione degli artefatti tecnologici che
rappresentano il sistema
– rappresentano come il sistema fa quello che deve fare
\section{Analysis}


Potenzialità software di Arduino UNO sfruttate:



Potenzialità hardware di Arduino UNO sfruttate:
\begin{itemize}
	\item sono stati utilizzati tutti i 12 pin di I/O digitale;
	\item si è cercato di limitare l'utilizzo di \textbf{delay} per mantenere le prestazioni ottimali;
	\item in alcuni casi al posto dei \textit{delay} si è fatto ricorso a dei \textbf{custom timer} impiegando il metodo \textbf{millis()};
\end{itemize}